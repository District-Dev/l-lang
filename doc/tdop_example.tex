\documentclass[beamer]{standalone}
% Format presentation size to A4
\setlength{\paperwidth}{29.7cm}
\setlength{\paperheight}{21.0cm}
\setlength{\textwidth}{28.7cm}
\setlength{\textheight}{20.0cm} 


%\def\pgfsysdriver{pgfsys-tex4ht.def} % for htlatex
\usepackage{tikz}
\usepackage{caption}
%\usepackage{ifthen}
\usepackage{etoolbox}

\usetikzlibrary{positioning}
\usetikzlibrary{shapes}
\usetikzlibrary{arrows}

% One way to do the animation is convert into animated gif (see
% example at http://www.texample.net/tikz/examples/eratosthenes-sieve/):
% pdfcrop tdop_example.pdf 
% convert -verbose -delay 100 -loop 0 -density 150 tdop_example-crop.pdf tdop_example.gif
% Then compress the animated gif:
% http://tex.stackexchange.com/questions/107382/package-animate-is-great-but-could-it-have-a-better-animation-compression/107508#107508
% But animed gif are too fast, no way to read them quietly.

% Instead, we convert to multiple png files: (order of argument is important)
% convert -density 600 -quality 90 tdop_example-crop.pdf tdop_example.png

% And we display the images using bootstrap "carousel"
% Afficher une image sur click:
% http://forum.hardware.fr/hfr/Programmation/HTML-CSS-Javascript/resolu-changement-image-sujet_76962_1.htm
% http://stackoverflow.com/questions/7916087/javascript-change-a-picture-when-clicking

% An animation, or static version
\newtoggle{isanimated}
%\toggletrue{isanimated}
\togglefalse{isanimated}

% Display the state of the stream on the edges of the tree
\newtoggle{displaystringinline}
\settoggle{displaystringinline}{\iftoggle{isanimated}{false}{true}}

\begin{document}
\pagestyle{empty}


\begin{standaloneframe}[fragile]
\begin{figure}[htbp]
%  \centering
\hspace{-2.5cm}\begin{tikzpicture}[xscale=3.3,yscale=1.4,scale=\iftoggle{displaystringinline}{2}{1.3}]

  \tikzstyle{node}=[rectangle, rounded corners,fill=red!40,draw=black];
  \tikzstyle{token}=[ellipse,fill=blue!40,draw=black];
  \tikzstyle{prefixtoken}=[ellipse,fill=green!40,draw=black];
  \tikzstyle{finaltoken}=[ellipse, fill=green!40,draw=black];
  \tikzstyle{fleche}=[-latex',shorten >=1.5pt, shorten <=1.5pt];

  \iftoggle{displaystringinline}{
  \newcommand{\edgereturn}[3]{to[bend right=5,#1] node[draw,rectangle,inner sep=0pt,#3 right=1mm] {\small $\!\!\begin{array}{l}#2\\ \texttt{"\texttoparse"}\end{array}\!\!$}}
  \newcommand{\edgetoinfix}[3]{to[bend right=5,#1] node[draw,rectangle,inner sep=0pt,#3 left=1mm] {\small $\!\!\begin{array}{r}#2\\ \texttt{"\texttoparse"}\end{array}\!\!$}}
  \newcommand{\edgetoprefix}[2]{to[bend right=5,#1] node[draw,rectangle,inner sep=0pt,#2 left=1mm] {\small $\!\!\begin{array}{r}\\ \texttt{"\texttoparse"}\end{array}\!\!$}}
}
{
  \newcommand{\edgereturn}[3]{to[bend right=5,#1] node[#3 right=1mm] {\small #2}}
  \newcommand{\edgetoinfix}[3]{to[bend right=5,#1] node[#3 left=1mm] {\small #2}}
  \newcommand{\edgetoprefix}[2]{to[bend right=5,#1]}
}


  \newcommand{\eof}{} % <eof>

  \newcommand{\texttoparse}{toto}
  \newcommand<>{\changetext}[1]{\only#2\renewcommand{\texttoparse}{#1}}


  \iftoggle{isanimated}{
    % For pdfcrop
  \fill[black!1] (-0.5,-5.2) rectangle (2.5,1.2);
  }{
    \draw[white] (0,1.2) -- (0,-5.2);
    \draw[white] (-0.5,1) -- (2.5,1);
  }

  \node (start) at (1,1) {start};
  \node<+->[node] (parse1) at (1,0) {parse(rbp=0)};
  \changetext<.->{-a+b*c+d}
  \draw<.->[fleche] (start) \edgetoprefix{midway}{} (parse1);

  \node<+->[prefixtoken] (-handler) at (0,-1) {$\mathtt{-}^3$};
  \changetext<.->{\textcolor{red}{-}a+b*c+d}
  \draw<.->[fleche] (parse1) \edgetoprefix{near end}{above} (-handler);

  \node<+->[node] (parse2) at (0,-2) {parse(rbp=3)};
  \changetext<.->{a+b*c+d}
  \draw<.->[fleche] (-handler) \edgetoprefix{midway}{} (parse2);
  \node<+->[finaltoken] (a) at (0,-3) {a};
  \changetext<.->{\textcolor{red}{a}+b*c+d}
  \draw<.->[fleche] (parse2) \edgetoprefix{midway}{} (a);

  \changetext<+->{+b*c+d}
  \draw<.->[fleche] (a) \edgereturn{midway}{a}{} (parse2);
  \changetext<+->{\textcolor{red}{+}b*c+d}
  \draw<.->[fleche] (parse2) \edgereturn{midway}{a}{} (-handler);
  \changetext<+->{+b*c+d}
  \draw<.->[fleche] (-handler) \edgereturn{near start}{-a}{below} (parse1);

  \node<+->[token] (+handler) at (1,-1) {${}^{1}\mathtt{+}^{1}$};
  \changetext<.->{\textcolor{red}{+}b*c+d}
  \draw<.->[fleche] (parse1) \edgetoinfix{near end}{-a}{} (+handler);
  \node<+->[node] (parse3) at (1,-2) {parse(rbp=1)};
  \changetext<.->{b*c+d}
  \draw<.->[fleche] (+handler) \edgetoprefix{midway}{} (parse3);
  \node<+->[finaltoken] (b) at (0.67,-3) {b};
  \changetext<.->{\textcolor{red}{b}*c+d}
  \draw<.->[fleche] (parse3) \edgetoprefix{near end}{above} (b);
  \changetext<+->{*c+d}
  \draw<.->[fleche] (b) \edgereturn{near start}{b}{below} (parse3);

  \node<+->[token] (*handler) at (1.33,-3) {${}^2\mathtt{*}^{2}$};
  \changetext<.->{\textcolor{red}{*}c+d}
  \draw<.->[fleche] (parse3) \edgetoinfix{near end}{b}{below} (*handler);
  \node<+->[node] (parse4) at (1.33,-4) {parse(rbp=2)};
  \changetext<.->{c+d}
  \draw<.->[fleche] (*handler) \edgetoprefix{midway}{} (parse4);
  \node<+->[finaltoken] (c) at (1.33,-5) {c};
  \changetext<.->{\textcolor{red}{c}+d}
  \draw<.->[fleche] (parse4) \edgetoprefix{midway}{} (c);
  \changetext<+->{+d}
  \draw<.->[fleche] (c) \edgereturn{midway}{c}{} (parse4);
  \changetext<+->{\textcolor{red}{+}d}
  \draw<.->[fleche] (parse4) \edgereturn{midway}{c}{} (*handler);
  \changetext<+->{+d}
  \draw<.->[fleche] (*handler) \edgereturn{near start}{b*c}{above} (parse3);
  \changetext<+->{\textcolor{red}{+}d}
  \draw<.->[fleche] (parse3) \edgereturn{midway}{b*c}{} (+handler);
  \changetext<+->{+d}
  \draw<.->[fleche] (+handler) \edgereturn{near start}{(-a)+(b*c)}{} (parse1);

  \node<+->[token] (+handler2) at (2,-1) {${}^{1}\mathtt{+}^{1}$};
  \changetext<.->{\textcolor{red}{+}d}
  \draw<.->[fleche] (parse1) \edgetoinfix{very near end}{(-a)+(b*c)}{below} (+handler2);
  \node<+->[node] (parse5) at (2,-2) {parse(rbp=1)};
  \changetext<.->{d}
  \draw<.->[fleche] (+handler2) \edgetoprefix{midway}{} (parse5);
  \node<+->[finaltoken] (d) at (2,-3) {d};
  \changetext<.->{\textcolor{red}{d}}
  \draw<.->[fleche] (parse5) \edgetoprefix{midway}{} (d);
  \changetext<+->{\eof}
  \draw<.->[fleche] (d) \edgereturn{midway}{d}{} (parse5);
  \draw<+->[fleche] (parse5) \edgereturn{midway}{d}{} (+handler2);
  \draw<+->[fleche] (+handler2) \edgereturn{near start}{((-a)+(b*c))+d}{above} (parse1);
  \draw<+->[fleche] (parse1) \edgereturn{midway}{((-a)+(b*c))+d}{} (start);

  \node[left] at (2,1) {\Huge \texttoparse};

\end{tikzpicture}

% \iftoggle{isanimated}{}{
% \caption*{TDOP is best understood through an example. This Figure represents the call tree of a TDOP parser for the expression \texttt{"-a+b*c+d"}. The order of execution is depth-first, with children visited from left to right.

%   The \texttt{parse} function is in red, prefix handlers are in green,
%   and infix handlers in blue. The first token called by the
%   \texttt{parse} function is in prefix position, the others are in
%   infix position. The left binding power and right binding power, when
%   applicable, are given as superscripts. The first token in the stream
%   is in red when it was just peeked by the \texttt{parse} function.}}

\end{figure}
\end{standaloneframe}
\end{document}
